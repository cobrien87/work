\documentclass{exam}

\pagenumbering{gobble}
\usepackage{titling}
\setlength{\droptitle}{-8ex}
\pretitle{\begin{flushleft}\Large\bfseries}
\posttitle{\par\end{flushleft}}
\preauthor{\begin{flushleft}\Large}
\postauthor{\end{flushleft}}
\predate{\begin{flushleft}}
\postdate{\end{flushleft}}
\usepackage{enumerate} %for alphabetized lists
\usepackage{amsmath}
\usepackage{multicol} %for multiple columns
\usepackage{systeme} %systems of equations
\renewcommand{\questionshook}{\setlength{\itemsep}{15pt}} %controls space betweenites

%fromhttps://texblog.org/2012/06/25/adding--lines--for--taking--handwritten--notes--in--latex/
\usepackage{pgffor, ifthen}
\newcommand{\notes}[3][\empty]{%
   vspace{10pt}\\
    \foreach \n in {1,...,#2}{%
        \ifthenelse{\equal{#1}{\empty}}
            {\rule{#3}{0.5pt}\\}
            {\rule{#3}{0.5pt}\vspace{#1}\\} 
        }
}

\title{Precalculus Quiz \#1 (Retake): Spring 2022  }

\author{  Name:   }
\begin{document}
\maketitle
\thispagestyle{empty}

\begin{questions}
\question 

\begin{multicols}{2}
A system of equations with an infinite number of solutions is... 

\columnbreak
\begin{choices}
\choice inconsistent.
\choice consistent.
\choice asymmetric.
\choice impossible.
\end{choices}

\end{multicols}



\question
\begin{multicols}{2}
A \textbf{coefficient matrix} will always contain...

\columnbreak

\begin{choices}
\choice exactly three columns.
\choice  more columns than variables.
\choice one row for every equation.
\choice  one column for every equation.
\end{choices}
\end{multicols}


\question
\begin{multicols}{2}
The system of equations below has an infinite number of solutions:

$
\systeme{
3x + 2y + z = 8,
-6x + 2z = 4
}
$

Which of the following is \textbf{not} a possible solution?

\columnbreak

\begin{choices}
\choice $x=1, y=0,z=5$
\choice $ x=0, y=3, z=2$
\choice $x=1,y=1,z=4$
\choice$ x=-1,y=6,z=-1$
\end{choices}
\end{multicols}
\question 
\label{two}

\begin{multicols}{2}
\hspace{1cm}$

\begin{bmatrix}
1 & -3 \\
0 & 0 \\
5 & -3
\end{bmatrix}
+
\begin{bmatrix}
2 & 3 &  14\\
0 & 0  & 0 \\

\end{bmatrix}
=
$
%\columnbreak

\vspace{1cm}
\begin{choices}
\choice 
$
\begin{bmatrix}
-1 & 3 & 11\\
1 & 0 & 5
\end{bmatrix}
$

\choice 
$
\begin{bmatrix}
3 & 3 & 19\\
-3 & 0 & -3
\end{bmatrix}
$

\choice 
$
\begin{bmatrix}
7 & 3 & 15\\
-3 & 0 & -3
\end{bmatrix}
$

\choice Matrix addition is undefined here.
\end{choices}

\columnbreak
Show your work or explain your answer here:
\makeemptybox{\stretch{1}}

\end{multicols}

\newpage






\question
\label{last}
\begin{multicols}{2}
For the system of equations to the right,
\begin{enumerate}[i.]
\item Convert to \textbf{augmented matrix} form
\item Use \textbf{Gaussian Elimination} to transform to row-echelon form
\item Solve for $x, y$ and $z$. For full credit use \textbf{Gauss-Jordan elimination}. For partial credit use back-substitition. 
\end{enumerate}



\columnbreak
$
\systeme{
x + 3y + 2z = 2,
2x + 7y + 7z = -1,
2x + 5y + 2z = 7
}
$


\end{multicols}

\makeemptybox{\stretch{1}}

\end{questions}

\newpage
\textbf{Use this space to continue work on (\ref{last}).}

\thispagestyle{empty}


\end{document}