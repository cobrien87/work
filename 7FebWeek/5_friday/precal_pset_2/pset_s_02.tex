\documentclass{exam}

\pagenumbering{gobble}
\usepackage{titling}
\setlength{\droptitle}{-8ex}
\pretitle{\begin{flushleft}\Large\bfseries}
\posttitle{\par\end{flushleft}}
\preauthor{\begin{flushleft}\Large}
\postauthor{\end{flushleft}}
\predate{\begin{flushleft}}
\postdate{\end{flushleft}}
\usepackage{enumerate} %for alphabetized lists
\usepackage{amsmath}
\usepackage{multicol} %for multiple columns

%fromhttps://texblog.org/2012/06/25/adding--lines--for--taking--handwritten--notes--in--latex/
\usepackage{pgffor, ifthen}
\newcommand{\notes}[3][\empty]{%
   vspace{10pt}\\
    \foreach \n in {1,...,#2}{%
        \ifthenelse{\equal{#1}{\empty}}
            {\rule{#3}{0.5pt}\\}
            {\rule{#3}{0.5pt}\vspace{#1}\\}
        }
}

\title{Pset\#2: Solving systems of equations with Gaussian Elimination }

\author{Precalculus \\ Spring Semester 2022 \\ Herbert H. Lehman High School}
\begin{document}
\maketitle
\noindent\textbf{Be sure to: } Complete all work in your notebook. Upload a photo to Google Classroom to submit.  Show all work!
\vspace{5mm}





\begin{questions}
\question

\textbf{Consider the following system of equations:}\\
\begin{center}
$
\begin{tabular}{r c r c r c r }
x & + & y & + & z & = & 48 \\

--x & + & 2y & + &2z & = & -24  \\

2x & -- & 6y & + & 4z & = & 12 \\
\end{tabular}
$
\end{center}
\vspace{0.5cm}
\begin{parts}
\part
\textbf{Steps (i)--(iii) below illustrate the conversion of this system into row--echelon form.  For each step, explain how it was derived from the previous step.  Be as explicit as possible!}
\vspace{0.3cm}

\begin{enumerate}[i.]

\item

\hspace{1cm}
$
\begin{tabular}{r c r c r c r }
x & + & y & + & z & = & 48 \\

 &  & 3y & + &3z & = & 24  \\

2x & -- & 6y & + & 4z & = & 12 \\
\end{tabular}
$
\vspace{0.5cm}
\item
\hspace{1cm}
$
\begin{tabular}{r c r c r c r }
x & + & y & + & z & = & 48 \\

 &  & 3y & + &3z & = & 24  \\

&  & 8y & -- & 2z & = & 84 \\
\end{tabular}
$
\vspace{0.5cm}
\item
\hspace{1cm}
$
\begin{tabular}{r c r c r c r }
x & + & y & + & z & = & 48 \\

 &  & 3y & + &3z & = & 24  \\

&  &  &  & 30z & = & -60 \\
\end{tabular}
$
\vspace{0.5cm}


\end{parts}
\part
\textbf{
Use back--substitution to solve for $x$, $y$, and $z$. Show all work.
} 
\part
\textbf{Finally, check the solution on the answer key (Ask Dr. O'Brien). If you got anything wrong, write a sentence explaining what you did wrong.}

\end{parts}
\newpage
 \uplevel{\textbf{For questions (2-4)} 
 
 \begin{enumerate}[i.]
 \item Use Gaussian elimination to convert the system  to row-echelon form.
 \item Use back substitution to solve for $x$, $y$, and $z$.
 
\item Finally, check your answer on  the answer key (Ask Dr. O'Brien).  If you got the wrong answer, write a sentence explaining what you did wrong.
 
 \end{enumerate}
 }


\begin{multicols}{2}
\question
\text{ }\\
$
\begin{tabular}{r c r c r c r }
x & + & y & --& z & = & -2 \\

2x & -- & y & + & z & = & 5  \\

-x & + & 2y & + & 2z & = & 1 \\
\end{tabular}
$
\columnbreak
\question
\text{ }\\
$
\begin{tabular}{r c r c r c r }
2x & + & 4y & --& 2z & = & 2 \\

4x & + & 9y & -- & 3z & = & 8  \\

-2x & -- & 3y & + & 7z & = & 10 \\
\end{tabular}
$
\end{multicols}
\vspace{0.5cm}
\question
\text{ }\\
$
\begin{tabular}{r c r c r c r }
x & + & y & --& 3z & = & -1 \\

 &  & y & -- & z & = & 0  \\

-x & + & 2y &  &  & = & 1 \\
\end{tabular}
$
\question
Are the two systems below equivalent? Be sure to explain  in a complete sentence, providing reasons for your answer:
\begin{multicols}{2}
\begin{parts}
\part
\begin{tabular}{r c r c r c r }
x & + & 3y & --& z & = & 6 \\

2x  &  --& y & +& 2z & = & 1  \\

3x & + & 2y & -- &  z& = & 2 \\
\end{tabular}
$

\part

\begin{tabular}{r c r c r c r }
x & + & 3y & --& z & = & 6 \\

  &  & -7y & +& 4z & = & 11  \\

 &  & -7y & -- &  4z& = & -16 \\
\end{tabular}
$
\end{parts}
\end{multicols}
\end{questions}
\end{document}