\documentclass{exam}

\pagenumbering{gobble}
\usepackage{titling}
\setlength{\droptitle}{-8ex}
\pretitle{\begin{flushleft}\Large\bfseries}
\posttitle{\par\end{flushleft}}
\preauthor{\begin{flushleft}\Large}
\postauthor{\end{flushleft}}
\predate{\begin{flushleft}}
\postdate{\end{flushleft}}
\usepackage{enumerate} %for alphabetized lists
\usepackage{amsmath}
\usepackage{multicol} %for multiple columns

%fromhttps://texblog.org/2012/06/25/adding--lines--for--taking--handwritten--notes--in--latex/
\usepackage{pgffor, ifthen}
\newcommand{\notes}[3][\empty]{%
   vspace{10pt}\\
    \foreach \n in {1,...,#2}{%
        \ifthenelse{\equal{#1}{\empty}}
            {\rule{#3}{0.5pt}\\}
            {\rule{#3}{0.5pt}\vspace{#1}\\}
        }
}

\title{Pset\#1: Solving systems of equations with Gaussian Elimination }

\author{Precalculus \\ Spring Semester 2022 \\ Herbert H. Lehman High School}
\begin{document}
\maketitle
\noindent\textbf{Be sure to: } Complete all work in your notebook. Upload a photo to Google Classroom to submit.  Show all work!
\vspace{5mm}





\begin{questions}
\question

\textbf{Consider the following system of equations:}

$
\begin{tabular}{r c r c r c r }
x & + & y & + & z & = & 48 \\

--x & + & 2y & + &2z & = & -24  \\

2x & -- & 6y & + & 4z & = & 12 \\
\end{tabular}
$

\textbf{Steps (a)--(c) below illustrate the conversion of this system into row--echelon form.  For each step, explain how it was derived from the previous step.  Be as explicit as possible!}
\vspace{0.3cm}
\begin{parts}

\part

\hspace{1cm}
$
\begin{tabular}{r c r c r c r }
x & + & y & + & z & = & 48 \\

 &  & 3y & + &3z & = & 24  \\

2x & -- & 6y & + & 4z & = & 12 \\
\end{tabular}
$
\vspace{0.5cm}
\part
\hspace{1cm}
$
\begin{tabular}{r c r c r c r }
x & + & y & + & z & = & 48 \\

 &  & 3y & + &3z & = & 24  \\

&  & 8y & -- & 2z & = & 84 \\
\end{tabular}
$
\vspace{0.5cm}
\part
\hspace{1cm}
$
\begin{tabular}{r c r c r c r }
x & + & y & + & z & = & 48 \\

 &  & 3y & + &3z & = & 24  \\

&  &  &  & 30z & = & -60 \\
\end{tabular}
$
\vspace{0.5cm}


\end{parts}
\textbf{
Finally, use back--substitution to solve for $x$, $y$, and $z$. 
} 

\question
\textbf{Use Gaussian eliminitation to convert the system below to row-echelon form, the use back substitution to solve:}
$
\begin{tabular}{r c r c r c r }
x & + & y & --& z & = & --2 \\

2x & -- & y & + & z & = & 5  \\

-x & + & 2y & + & 2z & = & 1 \\
\end{tabular}
$

\question
$
\begin{tabular}{r c r c r c r }
x & -- & 3y & +& z & = & 1 \\

2x & -- & y & -- & 2z & = & 2  \\

x & + & 2y & -- & 3z & = & -1 \\
\end{tabular}
$
\end{questions}
\end{document}