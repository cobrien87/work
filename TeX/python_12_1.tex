
%AIM AND CLASS!!!!
\newcommand{\aim}{MP1 recap/MP2 preview} 
\newcommand{\class}{Python }
\newcommand{\num}{ 2.2}
\newcommand{\reminders}{

\begin{enumerate}
	\item Weekly CodeHS expectations  up on Google Classroom
	\item  Only two marking periods this semester! 2nd marking period ends \textbf{late January}.
	\item \textbf{Daily Office Hours:} 2:36-3:06 (via Google Meet; see Google Classroom!)
	\item Assessment \#2 up on CodeHS!
\end{enumerate}
}
%for Python code
\documentclass{beamer}

% \usepackage{beamerthemesplit} // Activate for custom appearance

\title{Example Presentation Created with the Beamer Package}
\author{Till Tantau}
\date{\today}

\begin{document}

\frame{\titlepage}

\section[Outline]{}
\frame{\tableofcontents}

\section{Introduction}
\subsection{Overview of the Beamer Class}
\frame
{
  \frametitle{Features of the Beamer Class}

  \begin{itemize}
  \item<1-> Normal LaTeX class.
  \item<2-> Easy overlays.
  \item<3-> No external programs needed.      
  \end{itemize}
}
\end{document}


%\documentclass[t,handout,usenames,dvipsnames]{beamer}
\documentclass[t,aspectratio=169,usenames,dvipsnames]{beamer}

\usepackage{listings}

\usepackage{pgfpages}
\mode<handout>{%
	\pgfpagesuselayout{4 on 1}[a4paper] 
	\setbeameroption{show notes}
}


\usepackage{tcolorbox} %for colored boxes
\usepackage{framed}
\usepackage{multicol}
\usepackage{listings} %To insert code into presentation.
\lstset{escapeinside={<@}{@>}}

\usepackage{color}
\definecolor{lightgray}{rgb}{.9,.9,.9}
\definecolor{darkgray}{rgb}{.4,.4,.4}
\definecolor{purple}{rgb}{0.65, 0.12, 0.82}

\definecolor{mygreen}{rgb}{0,0.6,0}
\definecolor{mygray}{rgb}{0.5,0.5,0.5}
\definecolor{mymauve}{rgb}{0.58,0,0.82}

\lstset{ %
	backgroundcolor=\color{white},   % choose the background color
	basicstyle=\footnotesize,        % size of fonts used for the code
	breaklines=true,                 % automatic line breaking only at whitespace
	captionpos=b,                    % sets the caption-position to bottom
	commentstyle=\color{mygreen},    % comment style
	escapeinside={\%*}{*)},          % if you want to add LaTeX within your code
	keywordstyle=\color{blue},       % keyword style
	stringstyle=\color{mymauve},     % string literal style
}




\newcommand{\vb}{\vspace{1cm}}
\newcommand{\vs}{\vspace{0.5cm}}

\setbeamertemplate{definitions}[numbered] %numbered definitions

\usepackage[english]{babel}


\usepackage{times}
\usepackage[T1]{fontenc}
\newcommand{\R}{\mathbb{R}}
\newcommand{\C}{\mathbb{C}}


%HEADER GOES HERE
\setbeamertemplate{headline}{
	\hfill\includegraphics[width=1.5cm]{/home/chris/Dropbox/GEO/images/lehmanlogo.jpg}\hspace{0.2cm}\vspace{-1cm}}



%FOOTER GOES HERE
\setbeamerfont{footline}{size=\fontsize{13}{15}\selectfont}
\setbeamertemplate{footline}[text line]{%
  \parbox{\linewidth}{\vspace*{-8pt}\alert{Aim:} \aim \\ \alert{Date:} \today \hfill\insertshortauthor}}
\setbeamertemplate{navigation symbols}{}

\title[O'Brien Lesson plan notes] % (optional, use only with long paper titles)
{\class lesson \num}


\subtitle
{ }


\author[] % (optional, use only with lots of authors)
{Dr. O'Brien \\ Fall 2020}

\institute[Herbert H. Lehman High School] % (optional, but mostly needed)
{
  %\inst{1}%
  Herbert H. Lehman High School \\ Mr. Powers, Principal}

\date{ }







\begin{document}

\begin{frame}
  \titlepage
  
  \begin{center}
\includegraphics[width=2cm]{/home/chris/Dropbox/GEO/images/lehmanlogo.jpg}
\end{center}
\note{Preplanned questions are indicated by a (+). The anticipated response immediately follows, on the same line or the next line.
Directions indicated by a (@). }
\end{frame}



% DO NOW  DO NOW DO NOW
\begin{frame}
\frametitle<presentation>{ Do Now}
Answer in \textbf{Google Form} on Google Classroom
\begin{tcolorbox}[colback=blue!5,colframe=red!40!black ]
What do you think is  the most important thing you learned in Python during MP1? Explain why in a full paragraph.
\end{tcolorbox}
%NOTES NOTES NOTES
\note{...}
\end{frame}






\begin{frame}
\frametitle<presentation>{Announcements}
\vspace{-1cm}
\hspace{4cm}
\includegraphics[width=5cm]{/home/chris/Dropbox/GEO/images/sovcon.jpg}
\reminders

\note{...}
\end{frame}

\begin{frame}
\frametitle<presentation>{Weekly CodeHS Expectations}
\includegraphics[width=12cm]{expectations.png}
\note{...}
\end{frame}


\begin{frame}
\frametitle<presentation>{MP1 Recap}
Main themes last marking period:

\begin{itemize}
	\item<1-> Basic syntax for commands (e.g. \texttt{forward(50)})
	\item Using comments to describe your code
	\item Syntax for \textbf{for loops} and \textbf{functions}
	\item Using \textbf{for loops} and \textbf{functions} to express patterns in our code
\end{itemize}
\note{...}
\end{frame}


\begin{frame}[fragile]
\frametitle<presentation>{Expressing patterns in Python (1)}
\begin{enumerate}
	\item What pattern do you notice in \alert{example A}?
	\item How does \textcolor{blue}{example B} \textit{express} this pattern better?

\end{enumerate}
	

\begin{minipage}[t]{0.45\textwidth}
	\textcolor{red}{Example A}
	\begin{lstlisting}[language=Python]
	forward(50)
	left(90)
	forward(50)
	left(90)
	forward(50)
	left(90)
	forward(50)
	left(90)
	\end{lstlisting}
\end{minipage}
\begin{minipage}[t]{0.45\textwidth}
	\textcolor{blue}{Example B}
	\begin{lstlisting}[language=Python]
	for i in range(4):
	  forward(50)
	  left(90)
	\end{lstlisting}
\end{minipage}

\note{}
\end{frame}

\begin{frame}[plain, fragile]
\frametitle<presentation>{Expressing patterns in Python (2)}

\begin{enumerate}
	\item What pattern do you notice in \alert{example A}?
	\item How does \textcolor{blue}{example B} \textit{express} this pattern better?
	
\end{enumerate}




\begin{columns}[T]
\column{0.45\textwidth}
%\textcolor{red}{Example A}
\begin{tcolorbox}[colback=white!5,colframe=red!40!black,title=Example A ]
\begin{lstlisting}[language=Python]
for i in range(2):
  for i in range(4):
    forward(50)
    left(90)
  penup()
  forward(50)
  pendown()
  for i in range(4):
    forward(50)
    left(90)
	
\end{lstlisting}
\end{tcolorbox}
\column{0.5\textwidth}


\begin{tcolorbox}[colback=white!5,colframe=blue!40!black,title=Example B ]
	\begin{lstlisting}[language=Python]
	def square():
	  for i in range(4):
	    forward(50)
	    left(90)
	for i in range(2):
	  square()
	  penup()
	  forward(50)
	  pendown()
	  square()
	
	
	\end{lstlisting}
\end{tcolorbox}
\end{columns}


\note{...}
\end{frame}

\begin{frame}
\frametitle<presentation>{MP2 preview}
\begin{itemize}
	\item<1-> Top-down design and computational thinking
	\item<2-> Variables
	\item<3-> Function parameters
	\item<4-> Control structures: if(/else) statements, while loops
\end{itemize}
\note{...}
\end{frame}


 \begin{frame}
\frametitle<presentation>{ Logging into Zoom }
Use this link: https://us02web.zoom.us/j/82881041991?pwd=M0JuSmFWc0VuUi9NbVRCTHlveG1YUT09

Sign in with \textbf{Lehman email credentials}



\begin{tcolorbox}[colback=blue!5,colframe=red!40!black,title=Reminders ]
\reminders
\end{tcolorbox}

\note{...}
\end{frame}




\end{document}


\includegraphics[width=9cm]{exitticket.png}



\begin{columns}[T]
\column{0.5\textwidth}
\end{columns}



\begin{frame}
\frametitle<presentation>{Title here}
\note{...}
\end{frame}



 \begin{minipage}[t]{0.45\textwidth}
 \textcolor{red}{Example A}
 \begin{lstlisting}[language=Javascript]
 function <@\texttt{\textcolor{blue}{start}}@>() {

 }
 \end{lstlisting}
 \end{minipage}




\begin{frame}
\frametitle<presentation>{Split Do Now: Where do I go?}

\fbox{
\begin{minipage}[t]{0.45\textwidth}
\textcolor{BrickRed}{Group A}: \textbf{Go to the Do now Area. Open binder. Begin Do Now.}\\

Everyone else.

\end{minipage}
}
\fbox{
\begin{minipage}[t]{0.45\textwidth}
\textcolor{Blue}{Group B}: \textbf{Go to your workstation. Login to CodeHS. Work at a  \\ volume 0.}\\




\end{minipage}
}
\note{Period 2. Post on front smartboard. direct students to find their name as they come in.}
\end{frame}



\begin{frame}
\frametitle<presentation>{Do Now:  glossary, etc.}
\begin{enumerate}
\item<1-> Log on to Google Classroom.
\item<2-> You'll find your \alert{Unit 1 glossary} in the \textbf{Classwork} tab.
\item<3-> Make sure you have entries for all these vocab. items!

\begin{tcolorbox}[colback=orange!5,colframe=orange!40!black,title=Vocabulary]
\begin{itemize}
\item
\alert{ }:

\item
\alert{ }:


\end{itemize}
\end{tcolorbox}
\item<4-> When your glossary is updated, log into CodeHS. Work on Karel/Javascript lessons.
\end{enumerate}




